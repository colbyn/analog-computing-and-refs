\newpage
\section*{References}

\newcommand{\sourceEntry}[2]{
\item [({#1})]{#2}
}

\begin{description}
\sourceEntry{Does Thinking Really Hard Burn More Calories?}{\url{https://www.scientificamerican.com/article/thinking-hard-calories/}}
\sourceEntry{NVIDIA A100
TENSOR CORE GPU}{\url{https://www.nvidia.com/content/dam/en-zz/Solutions/Data-Center/a100/pdf/nvidia-a100-datasheet.pdf}}
\sourceEntry{The Thermodynamics of Brains and Computers}{\url{https://webhome.phy.duke.edu/~hsg/363/table-images/brain-vs-computer.html}}
\sourceEntry{Sarpeshkar}{Sarpeshkar, Rahul. Analog Supercomputers: From Quantum Atom to Living Body, \url{https://youtu.be/ZycidN_GYo0}.}
\sourceEntry{Hehner}{Hehner, Eric \& Horspool, R.. (1979). A New Representation of the Rational Numbers for Fast Easy Arithmetic. SIAM J. Comput.. 8. 124-134. 10.1137/0208011.}
\sourceEntry{O'Donnell}{O'Donnell, Kevin. (2018). Digital computer simulation of an electronic analog computer. 1-7. 10.1109/LISAT.2018.8378025. }
\sourceEntry{Achour}{Achour, Sara \& Rinard, Martin. (2020). Noise-Aware Dynamical System Compilation for Analog Devices with Legno. 149-166. 10.1145/3373376.3378449.}
\sourceEntry{The AI Hardware Problem}{The AI Hardware Problem. YouTube, \url{https://youtu.be/owe9cPEdm7k}.}
\sourceEntry{Grand View Research, Inc}{Computational Biology Market Size Worth \$13.6 Billion By 2026: Grand View Research, Inc. \url{https://www.prnewswire.com/news-releases/computational-biology-market-size-worth-13-6-billion-by-2026-grand-view-research-inc-300865211.html}.}
\sourceEntry{Flamholz}{Flamholz, Avi et al. “The quantified cell.” Molecular biology of the cell vol. 25,22 (2014): 3497-500. doi:10.1091/mbc.E14-09-1347}
\sourceEntry{Trafton}{Anne Trafton (Massachusetts Institute of Technology), Cell-inspired electronics. \url{https://phys.org/news/2010-02-cell-inspired-electronics.html}.}
\sourceEntry{Tianhe-1}{Tianhe-1, Wikipedia, \url{https://en.wikipedia.org/wiki/Tianhe-1\#Tianhe-1A}.}
\sourceEntry{Bernard}{Dr. Bernard, A Future Of Medicine. Heme Review, \url{https://youtu.be/iVt5BpoTHYg}.}
\sourceEntry{Francesco}{Caravelli, Francesco, and Juan Carbajal. “Memristors for the Curious Outsiders.” Technologies 6.4 (2018): 118. Crossref. Web.}
\sourceEntry{Siegelmann}{Hava T. Siegelmann. 1999. Neural networks and analog computation: beyond the Turing limit. Birkhauser Boston Inc.}
\sourceEntry{Kish}{Kish, Laszlo. (2003). Quantum Computing with Analog Circuits: Hilbert Space Computing. Proceedings of SPIE - The International Society for Optical Engineering. 5055. 10.1117/12.497438. }
\sourceEntry{MacLennan}{MacLennan, Bruce (2007). A review of analog computing. Technical Report CS-07-601, Department of Electrical Engineering \& Computer Science. University of Tennessee, Knoxville}
\sourceEntry{Tucker}{John V. Tucker and Jeffery I. Zucker. 2007. Computability of analog networks. Theor. Comput. Sci. 371, 1–2 (February, 2007), 115–146. DOI:https://doi.org/10.1016/j.tcs.2006.10.018}
\sourceEntry{Sauro}{Sauro, H., Kim, K. It's an analog world. Nature 497, 572–573 (2013). https://doi.org/10.1038/nature12246}
\sourceEntry{Sarpeshkar}{Sarpeshkar, R. “Analog synthetic biology.” Philosophical transactions. Series A, Mathematical, physical, and engineering sciences vol. 372,2012 20130110. 24 Feb. 2014, doi:10.1098/rsta.2013.0110}
\sourceEntry{Kravitz}{Kravitz, Richard L et al. “Evidence-based medicine, heterogeneity of treatment effects, and the trouble with averages.” The Milbank quarterly vol. 82,4 (2004): 661-87. doi:10.1111/j.0887-378X.2004.00327.x}
\sourceEntry{Siegelmann}{Siegelmann, H T. “Computation beyond the turing limit.” Science (New York, N.Y.) vol. 268,5210 (1995): 545-8. doi:10.1126/science.268.5210.545}
\sourceEntry{Tyson}{Tyson, John \& Albert, Reka \& Goldbeter, Albert \& Ruoff, Peter \& Sible, Jill. (2008). Biological switches and clocks. Journal of the Royal Society, Interface / the Royal Society. 5 Suppl 1. S1-8. 10.1098/rsif.2008.0179.focus. }
\sourceEntry{Achour}{Achour, Sara \& Rinard, Martin. (2020). Noise-Aware Dynamical System Compilation for Analog Devices with Legno. 149-166. 10.1145/3373376.3378449. }
\end{description}

% \section*{Miscellaneous}
% \begin{center}
%     Source Code: \url{https://github.com/colbyn/engl-brainstorming-your-research-interests}.
% \end{center}

