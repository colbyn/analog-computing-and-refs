% 1. Author name(s), credentials, and background (briefly).
% 2. Type of source (report, study, peer-reviewed journal article, website, book, podcast episode, blog post, newspaper article, museum exhibition, etc.) and the main topic it addresses.
% 3. Summary of the source’s methodology: authors use primary research (empirical; qualitative or quantitative);  or, authors refer to secondary research and offer a meta-analysis; or, authors write an interpretive essay or article in which they draw on other authors and their ideas or concepts. 
% 4. The main thesis and/or the conclusions most pertinent to your research questions or project goals. Avoid quotations--paraphrase instead, providing correct in-text parenthetical citations. Use author tags (such as “According to X” or “As X suggests”) in every sentence of your summary to attribute these details to the author(s) of the source. 
% 5. Any limitations or biases in the source; consider the set-up of the primary research, or the study sample, or the scope of the research. Often, researchers will address some limitations themselves in their article. 
% 6. Evaluation of how you might use the source in your upcoming essays. Use first-person meta commentary (such as “I intend” or “I plan”) in every sentence of your evaluation to attribute these details to yourself, and not the author(s) of your source.


% 3. Summary of the source’s methodology: authors use primary research (empirical; qualitative or quantitative);  or, authors refer to secondary research and offer a meta-analysis; or, authors write an interpretive essay or article in which they draw on other authors and their ideas or concepts. 

\section*{Memristors for the Curious Outsiders}

Caravelli, Francesco, and Juan Carbajal. “Memristors for the Curious Outsiders.” Technologies 6.4 (2018): 118. Crossref. Web.

\begin{enumerate}
    \item Caravelli, Francesco, and Juan Carbajal.
    \item Peer-reviewed journal article.
    \item Authors refer to secondary research and offer a meta-analysis.
    \item An overview is given on a relatively new technology called ``Memristors''. Which has a number of interesting properties. For instance, from what I understand, such is immune to radiation, and is non-volatile (you can power up your computer without loss of information). It's forty-eight pages in total, and here, i'm only concerned with the latter part that pertains to analog computing. 
    \item None given. 
    \item I plan to use such in my analysis of analog computing. 
\end{enumerate}



\section*{Neural Networks and Analog Computation Beyond the Turing Limit}

Hava T. Siegelmann. 1999. Neural networks and analog computation: beyond the Turing limit. Birkhauser Boston Inc.

\begin{enumerate}
    \item Hava T. Siegelmann.
    \item Peer-reviewed journal article.
    \item Authors refer to secondary research and offer a meta-analysis.
    \item Discusses analog computation in the context of artificial neural networks.
    \item None given. 
    \item I plan to use such in my analysis of analog computing. 
\end{enumerate}

\section*{Quantum Computing with Analog Circuits: Hilbert Space Computing}

Kish, Laszlo. (2003). Quantum Computing with Analog Circuits: Hilbert Space Computing. Proceedings of SPIE - The International Society for Optical Engineering. 5055. 10.1117/12.497438. 

\begin{enumerate}
    \item Kish, Laszlo.
    \item Peer-reviewed journal article.
    \item Authors refer to secondary research and offer a meta-analysis.
    \item The author postulates that Quantum Computing may be implemented in a commercially viable manner as a classical analog computer. 
    \item None given. 
    \item I plan to use such in my analysis of analog computing. 
\end{enumerate}

\section*{A Review of Analog Computing}

MacLennan, Bruce (2007). A review of analog computing. Technical Report CS-07-601, Department of Electrical Engineering \& Computer Science. University of Tennessee, Knoxville

\begin{enumerate}
    \item MacLennan, Bruce.
    \item Peer-reviewed journal article.
    \item Authors refer to secondary research and offer a meta-analysis.
    \item The author gives an overview of analog computing.
    \item None given. 
    \item I plan to use such in my analysis of analog computing. 
\end{enumerate}

\section*{Computability of analog networks}

John V. Tucker and Jeffery I. Zucker. 2007. Computability of analog networks. Theor. Comput. Sci. 371, 1–2 (February, 2007), 115–146. DOI:https://doi.org/10.1016/j.tcs.2006.10.018

\begin{enumerate}
    \item John V. Tucker and Jeffery I. Zucker.
    \item Peer-reviewed journal article.
    \item Authors refer to secondary research and offer a meta-analysis.
    \item The authors discusses an implementation of analog computing with respect to a global continuous clock.
    \item None given. 
    \item I plan to use such in my analysis of analog computing. 
\end{enumerate}

\section*{It’s an analog world}
Sauro, H., Kim, K. It's an analog world. Nature 497, 572–573 (2013). https://doi.org/10.1038/nature12246

\begin{enumerate}
    \item John V. Tucker and Jeffery I. Zucker.
    \item Peer-reviewed journal article.
    \item Authors refer to secondary research and offer a meta-analysis.
    \item The authors discuss implementing functional circuits in living systems based on an analog model.
    \item None given. 
    \item I plan to use such in my analysis of analog computing. 
\end{enumerate}


\section*{Analog synthetic biology}

Sarpeshkar, R. “Analog synthetic biology.” Philosophical transactions. Series A, Mathematical, physical, and engineering sciences vol. 372,2012 20130110. 24 Feb. 2014, doi:10.1098/rsta.2013.0110

\begin{enumerate}
    \item John V. Tucker and Jeffery I. Zucker.
    \item Peer-reviewed journal article.
    \item Authors refer to secondary research and offer a meta-analysis.
    \item The author discusses analog computing in the context of systems biology, both in terms of encoding computation in biological systems, and from the perspective of simulating these very same systems electronically. Furthermore, the author postulates that there exists a deep connection between electrons and chemistry, in the sense that we can define an isomorphism (my own words) between the chemistry that manifests real life, and modeling the very same systems `electronically'.  
    \item None given. 
    \item I plan to use such in my analysis of analog computing. 
\end{enumerate}

To elaborate on this isomorphism, this is how the author put it,

\begin{quotation}
    ``electron concentration at the source is analogous to reactant concentration; electron concentration at the drain is analogous to product concentration; forward and reverse current flows in the transistor are analogous to forward and reverse reaction rates in a chemical reaction; the forward and reverse currents in a transistor are exponential in voltage differences at its terminals analogous to reaction rates being exponential in the free-energy differences in a chemical reaction; increases in gate voltage lower energy barriers in a transistor increasing current flow analogous to the effects of enzymes or catalysts in chemical reactions that increase reaction rates; and the stochastics of the Poisson shot noise in subthreshold transistors are analogous to the stochastics of molecular shot noise in reactions''.
\end{quotation}

Amount other things (the overall paper was twenty-two pages long).


\section*{Computation Beyond the Turing Limit}

Siegelmann, H T. “Computation beyond the turing limit.” Science (New York, N.Y.) vol. 268,5210 (1995): 545-8. doi:10.1126/science.268.5210.545

\begin{enumerate}
    \item Siegelmann, Hava.
    \item Peer-reviewed journal article.
    \item Authors refer to secondary research and offer a meta-analysis.
    \item This a short, and very difficult read. It's the kind of literature that I expect to one day understand, and therefore, make a personal note to review this at such a time. As is, note it's existence, in the chance that I happen to be working in the area, but I digress.
    
    In the paper, the author deposits a philosophical claim, that in humanities quest in both deciphering and mastering nature, has inevitably become a quest in building machines that can simulate such systems. In this paper, the author discusses simulating natural phenomena using a particular manifestation of analog computing, which is claimed to exceed the performance of todays computer architectures (presumably referring to the ``von neumann computer architecture''). 
    \item None given. 
    \item I plan to use such in my analysis of analog computing. 
\end{enumerate}



\section*{Biological switches and clocks}

Tyson, John \& Albert, Reka \& Goldbeter, Albert \& Ruoff, Peter \& Sible, Jill. (2008). Biological switches and clocks. Journal of the Royal Society, Interface / the Royal Society. 5 Suppl 1. S1-8. 10.1098/rsif.2008.0179.focus. 

\begin{enumerate}
    \item Tyson, John \& Albert, Reka \& Goldbeter, Albert \& Ruoff, Peter \& Sible, Jill.
    \item Peer-reviewed journal article.
    \item Authors refer to secondary research and offer a meta-analysis.
    \item To my understanding, only $1.5\%$ of your DNA encodes for proteins, while about $5\%$ of your DNA are regulatory sequences.
    
    The authors begin with a remark on the similarities between the information processing facilities of cells and that of manmade computers, and that, just as a ``sophisticated theory of electronic circuitry'' made modern manmade computers possible, so too, will a similar model of biomolecular circuitry give way to exploiting the regulatory mechanisms of cellular systems. 

    Later in, the authors discuss recent advancements from mathematical modelers in explaining such natural phenomena.
    \item None given. 
    \item I plan to use such in my analysis of analog computing. 
\end{enumerate}

