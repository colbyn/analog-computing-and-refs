\section*{Bibliography}

\subsection*{A Future of Medicine}
Dr. Bernard, A Future Of Medicine. Heme Review, \url{https://youtu.be/iVt5BpoTHYg}.
\begin{enumerate}
    \item Caravelli, Francesco, and Juan Carbajal.
    \item Peer-reviewed journal article.
    \item The author writes an interpretive video essay in which he draws upon prior research and events.
    \item In this essay, Dr. Bernard explains how our current medial system works and some more ideal hypothetical system, respectively. What is important, and the thesis of his essay, are the flaws of our current medical system (which he explains in exquisite detail). Overall, what is interesting -I think- is how our modern evidence based system isn’t as ideal as I once thought, and this is due to it’s dependence on population data, so such may not be directly applicable to a given person. For example, if a trial was done on a bunch of men, the speaker said clinical experience suggests that there will be differences for women taking the same medication. Whereas, his proposed hypothetical system disregards the need for such models VIA personalized simulations of a given individual, and I suppose all their biological processes. I.e. if this technology was available, there would be no need to study the efficacy VIA population models, because with this hypothetical technology, such simulations would be far more accurate, would be far more personalized to the biochemistry of the given individual, and not from generalizations of population data. Furthermore, if possible, he said, such could likewise -hypothetically speaking- be extended to drug development VIA optimization algorithms. \textsuperscript{(Bernard)}
    \item None given. 
    \item I plan to use such in my analysis of analog computing. 
\end{enumerate}


\subsection*{Memristors for the Curious Outsiders}
Caravelli, Francesco, and Juan Carbajal. “Memristors for the Curious Outsiders.” Technologies 6.4 (2018): 118. Crossref. Web.
\begin{enumerate}
    \item Caravelli, Francesco, and Juan Carbajal.
    \item Peer-reviewed journal article.
    \item Authors refer to secondary research and offer a meta-analysis.
    \item An overview is given on a relatively new technology called ``Memristors'', and it's potential applications.
    
    What memristors bring to the table (among other things) is more efficient computation for certain application specific workloads. To explain, consider the most performant offering from the Google cloud A2 Accelerator-Optimized tier, the ``Nvidia A100 tensor core GPU'', which can perform $3.12$ trillion (four byte) operations per watt. While in contrast, the human brain can perform about $83.\bar{3}$ trillion operations per watt. So therefore, while akin to an apples to oranges comparison, there may be room for improvement for low precision workloads. In this case, memristors may be ideal for implementing artificial neural networks.
        \item None given. 
    \item I plan to use such in my analysis of analog computing. 
\end{enumerate}



\subsection*{Neural Networks and Analog Computation Beyond the Turing Limit}
Hava T. Siegelmann. 1999. Neural networks and analog computation: beyond the Turing limit. Birkhauser Boston Inc.
\begin{enumerate}
    \item Hava T. Siegelmann.
    \item Peer-reviewed journal article.
    \item Authors refer to secondary research and offer a meta-analysis.
    \item In this 192 page document, the author discusses analog computation in the context of artificial neural networks. (Which is the sort of technology that can build upon the aforementioned ``Memristors''). The author begins with a discussion on the shortcomings of the predominate von neumann computer architecture, which is a model where memory and computation are discrete and physically separate entities. 
    \item None given. 
    \item I plan to use such in my analysis of analog computing. 
\end{enumerate}

\subsection*{Quantum Computing with Analog Circuits: Hilbert Space Computing}
Kish, Laszlo. (2003). Quantum Computing with Analog Circuits: Hilbert Space Computing. Proceedings of SPIE - The International Society for Optical Engineering. 5055. 10.1117/12.497438. 
\begin{enumerate}
    \item Kish, Laszlo.
    \item Peer-reviewed journal article.
    \item Authors refer to secondary research and offer a meta-analysis.
    \item The author postulates that Quantum Computing may be implemented in a commercially viable manner as a classical analog computer. 
    \item None given. 
    \item I plan to use such in my analysis of analog computing. 
\end{enumerate}

\subsection*{A Review of Analog Computing}
MacLennan, Bruce (2007). A review of analog computing. Technical Report CS-07-601, Department of Electrical Engineering \& Computer Science. University of Tennessee, Knoxville
\begin{enumerate}
    \item MacLennan, Bruce.
    \item Peer-reviewed journal article.
    \item Authors refer to secondary research and offer a meta-analysis.
    \item The author gives an overview of analog computing.
    \item None given. 
    \item I plan to use such in my analysis of analog computing. 
\end{enumerate}

\subsection*{Computability of analog networks}
John V. Tucker and Jeffery I. Zucker. 2007. Computability of analog networks. Theor. Comput. Sci. 371, 1–2 (February, 2007), 115–146. DOI:https://doi.org/10.1016/j.tcs.2006.10.018
\begin{enumerate}
    \item John V. Tucker and Jeffery I. Zucker.
    \item Peer-reviewed journal article.
    \item Authors refer to secondary research and offer a meta-analysis.
    \item The authors discusses an implementation of analog computing with respect to a global continuous clock.
    \item None given. 
    \item I plan to use such in my analysis of analog computing. 
\end{enumerate}

\subsection*{It’s an analog world}
Sauro, H., Kim, K. It's an analog world. Nature 497, 572–573 (2013). https://doi.org/10.1038/nature12246
\begin{enumerate}
    \item John V. Tucker and Jeffery I. Zucker.
    \item Peer-reviewed journal article.
    \item Authors refer to secondary research and offer a meta-analysis.
    \item The authors discuss implementing functional circuits in living systems based on an analog model.
    \item None given. 
    \item I plan to use such in my analysis of analog computing. 
\end{enumerate}


\subsection*{Analog synthetic biology}
Sarpeshkar, R. “Analog synthetic biology.” Philosophical transactions. Series A, Mathematical, physical, and engineering sciences vol. 372,2012 20130110. 24 Feb. 2014, doi:10.1098/rsta.2013.0110
\begin{enumerate}
    \item John V. Tucker and Jeffery I. Zucker.
    \item Peer-reviewed journal article.
    \item Authors refer to secondary research and offer a meta-analysis.
    \item The author discusses analog computing in the context of systems biology, both in terms of encoding computation in biological systems, and from the perspective of simulating these very same systems electronically. Furthermore, the author postulates that there exists a deep connection between electrons and chemistry, in the sense that we can define an isomorphism (my own words) between the chemistry that manifests real life, and modeling the very same systems `electronically'.  
    \item None given. 
    \item I plan to use such in my analysis of analog computing. 
\end{enumerate}

\subsection*{Evidence-Based Medicine, Heterogeneity of Treatment Effects, and the Trouble with Averages}
Kravitz, Richard L et al. “Evidence-based medicine, heterogeneity of treatment effects, and the trouble with averages.” The Milbank quarterly vol. 82,4 (2004): 661-87. doi:10.1111/j.0887-378X.2004.00327.x
\begin{enumerate}
    \item Kravitz, Richard.
    \item Peer-reviewed journal article.
    \item Authors refer to secondary research and offer a meta-analysis.
    \item The authors discuss the difficulties of our modern evidence based system of medicine. This is, we rely on population models to predict outcome for a given patent. Or put another way, we make generalizations, and so therefore, we may ask, do these generalizations always hold up? In their paper, the authors essentially say no, due to a phenomena the authors call, the ``Heterogeneity of Treatment Effects ''. The authors later discuss the implications for relevant parties. 
    \item None given. 
    \item I plan to use such in my analysis of analog computing. 
\end{enumerate}

\subsection*{Computation Beyond the Turing Limit}
Siegelmann, H T. “Computation beyond the turing limit.” Science (New York, N.Y.) vol. 268,5210 (1995): 545-8. doi:10.1126/science.268.5210.545
\begin{enumerate}
    \item Siegelmann, Hava.
    \item Peer-reviewed journal article.
    \item Authors refer to secondary research and offer a meta-analysis.
    \item This a short, and very difficult read. It's the kind of literature that I expect to one day understand, and therefore, make a personal note to review this at such a time. As is, note it's existence, in the chance that I happen to be working in the area, but I digress.
    
    In the paper, the author deposits a philosophical claim, that in humanities quest in both deciphering and mastering nature, has inevitably become a quest in building machines that can simulate such systems. In this paper, the author discusses simulating natural phenomena using a particular manifestation of analog computing, which is claimed to exceed the performance of todays computer architectures (presumably referring to the ``Von neumann computer architecture''). 
    \item None given. 
    \item I plan to use such in my analysis of analog computing. 
\end{enumerate}



\subsection*{Biological switches and clocks}
Tyson, John \& Albert, Reka \& Goldbeter, Albert \& Ruoff, Peter \& Sible, Jill. (2008). Biological switches and clocks. Journal of the Royal Society, Interface / the Royal Society. 5 Suppl 1. S1-8. 10.1098/rsif.2008.0179.focus. 
\begin{enumerate}
    \item Tyson, John \& Albert, Reka \& Goldbeter, Albert \& Ruoff, Peter \& Sible, Jill.
    \item Peer-reviewed journal article.
    \item Authors refer to secondary research and offer a meta-analysis.
    \item To my understanding, only $1.5\%$ of your DNA encodes for proteins, while about $5\%$ of your DNA are regulatory sequences.
    
    The authors begin with a remark on the similarities between the information processing facilities of cells and that of manmade computers, and that, just as a ``sophisticated theory of electronic circuitry'' made modern manmade computers possible, so too, will a similar model of biomolecular circuitry give way to exploiting the regulatory mechanisms of cellular systems. 

    Later in, the authors discuss recent advancements from mathematical modelers in explaining such natural phenomena.
    \item None given. 
    \item I plan to use such in my analysis of analog computing. 
\end{enumerate}

\subsection*{Noise-Aware Dynamical System Compilation for Analog Devices with Legno}

Achour, Sara \& Rinard, Martin. (2020). Noise-Aware Dynamical System Compilation for Analog Devices with Legno. 149-166. 10.1145/3373376.3378449. 

\begin{enumerate}
    \item source: \url{https://people.csail.mit.edu/sachour/docs/asplos20-legno.pdf}.
\end{enumerate}

